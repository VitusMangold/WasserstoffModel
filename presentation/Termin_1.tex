% !TeX program = lualatex
\documentclass[11pt, ngerman, hyperref={unicode}]{beamer}

\usefonttheme{professionalfonts}
\usetheme{CambridgeUS}
\usecolortheme{rose}
\usepackage{fontspec}
\usepackage{unicode-math}
%\usepackage{luatextra}
%\defaultfontfeatures{Ligatures=TeX,Numbers=OldStyle}
\setmainfont{Latin Modern Sans}
\setsansfont{Latin Modern Sans}
\setmonofont{Menlo}
%\setmathfont{latinmodern-math.otf}
\setmathrm{Latin Modern Roman}

\usepackage[ngerman]{babel}

\usepackage{amsmath}
\usepackage{amssymb}
\usepackage{amsthm}
\setbeamertemplate{theorems}[numbered]
\newtheorem*{definition*}{Definition}
\usepackage{mathtools}
\usepackage{hyperref}
\usepackage{animate}
\usepackage{algorithm2e}[plain, noline, ngerman]
\usepackage{minted}
\usepackage{csquotes}
\usepackage{tikz}
\usepackage{pgfplots}
\usetikzlibrary{arrows.meta}
\usetikzlibrary{cd}
\usetikzlibrary{babel}
\usetikzlibrary{pgfplots.groupplots}
\usepgfplotslibrary{fillbetween}
\pgfplotsset{compat=1.16}
\usepgfplotslibrary{external} 
\tikzexternalize

\makeatletter
\newcommand\pgfinvisible{\pgfsys@begininvisible}
\newcommand\pgfshown{\pgfsys@endinvisible}
\makeatother

%\usepackage[charsperline=65]{jlcode}
\usepackage[backend=biber,style=verbose]{biblatex}
\addbibresource{./literatur.bib}
\newcounter{footcounter}
\setcounter{footcounter}{1}

\makeatletter
\def\@makefnmark{\hbox{{{\usebeamercolor[fg]{footnote mark}\usebeamerfont*{footnote mark} [\hyperlink{\thefootcounter}{\@thefnmark]}}}}}

\def\@makefntext#1{%
	\def\insertfootnotetext{ \hypertarget{\thefootcounter}{#1}}%
	\def\insertfootnotemark{\@makefnmark}%
	\usebeamertemplate***{footnote}\stepcounter{footcounter}}    
\makeatother
\AtEveryCitekey{\clearfield{url}} % keine url in footcite
\DeclareFieldFormat{title}{#1}
\DeclareFieldFormat{citetitle}{#1}

\setbeamertemplate{navigation symbols}{}
\setbeamertemplate{caption}[numbered]

\title{Wasserstoff für Deutschland}
\author[Mangold, Nägele, Radmer]{Vitus Mangold, Johannes Nägele, Lena-René Radmer}
%\date{May 07, 2022}
\date{\today}

\newcommand{\cA}{\mathcal{A}}
\newcommand{\cB}{\mathcal{B}}
\newcommand{\cC}{\mathcal{C}}
\newcommand{\cD}{\mathcal{D}}
\newcommand{\cE}{\mathcal{E}}
\newcommand{\cF}{\mathcal{F}}
\newcommand{\cG}{\mathcal{G}}
\newcommand{\cH}{\mathcal{H}}
\newcommand{\cI}{\mathcal{I}}
\newcommand{\cJ}{\mathcal{J}}
\newcommand{\cK}{\mathcal{K}}
\newcommand{\cL}{\mathcal{L}}
\newcommand{\cM}{\mathcal{M}}
\newcommand{\cN}{\mathcal{N}}
\newcommand{\cO}{\mathcal{O}}
\newcommand{\cP}{\mathcal{P}}
\newcommand{\cQ}{\mathcal{Q}}
\newcommand{\cR}{\mathcal{R}}
\newcommand{\cS}{\mathcal{S}}
\newcommand{\cT}{\mathcal{T}}
\newcommand{\cU}{\mathcal{U}}
\newcommand{\cV}{\mathcal{V}}
\newcommand{\cW}{\mathcal{W}}
\newcommand{\cX}{\mathcal{X}}
\newcommand{\cY}{\mathcal{Y}}
\newcommand{\cZ}{\mathcal{Z}}
%% math blackboard bold
\newcommand{\A}{\mathbb{A}}
\newcommand{\B}{\mathbb{B}}
%\newcommand{\C}{\mathbb{C}}
\newcommand{\D}{\mathbb{D}}
\newcommand{\E}{\mathbb{E}}
\newcommand{\F}{\mathbb{F}}
%\newcommand{\G}{\mathbb{G}}
\newcommand{\I}{\mathbb{I}}
\newcommand{\J}{\mathbb{J}}
\newcommand{\K}{\mathbb{K}}
\newcommand{\M}{\mathbb{M}}
\newcommand{\N}{\mathbb{N}}
\newcommand{\Q}{\mathbb{Q}}
\newcommand{\R}{\mathbb{R}}

%\newcommand{\U}{\mathbb{U}}
\newcommand{\V}{\mathbb{V}}
\newcommand{\W}{\mathbb{W}}
\newcommand{\X}{\mathbb{X}}
\newcommand{\Y}{\mathbb{Y}}
\newcommand{\Z}{\mathbb{Z}}
\newcommand{\dint}{\, \mathrm{d}}

\newcounter{slideIndex}
\setcounter{slideIndex}{1}

\pgfmathdeclarefunction{gauss}{2}{%
	\pgfmathparse{1/(#2*sqrt(2*pi))*exp(-((x-#1)^2)/(2*#2^2))}%
}

\pgfmathdeclarefunction{cauchy}{1}{%
	\pgfmathparse{1/((1+(x)^2)*pi)+#1}%
}

\ExplSyntaxOn
\DeclareExpandableDocumentCommand \round { O{0} m }
{ \fp_eval:n { round(#2,#1) } }
\ExplSyntaxOff

\theoremstyle{definition} % insert bellow all blocks you want in normal text
\newtheorem{theorem1}{Theorem}
\newtheorem*{definition1}{Definition}
\newtheorem*{remark}{Bemerkung}

\begin{document}
	\setbeamertemplate{title page}[default][colsep=-4bp,rounded=true]
	\maketitle
	\begin{frame}{Inhalt}
		\tableofcontents
	\end{frame}
	\section{Einführung und Intuition}
	\begin{frame}{Problemstellung I}
		Foo
		\begin{itemize}
			\item bar
			\item foobar
		\end{itemize}
	\end{frame}

	\section{Datenlage}

	\begin{frame}{Problemstellung I}
		Foo
		\begin{itemize}
			\item bar
			\item foobar
		\end{itemize}
	\end{frame}

	\section{Lastflussberechnung}

	\begin{frame}{Bespielcode}
		\vspace{-0.1cm}
		Definiere den AM-Algorithmus mit normalverteiltem Kern nach \footcite[247-248]{Givens2012}.
		\vspace{-0.1cm}
		\definecolor{LightBlue}{RGB}{220,220,255}
		\setbeamercolor{block title example}{fg = LightBlue, bg = LightBlue}
		\begin{exampleblock}{}
			\vspace{-0.1cm}
			\SetAlgoNoLine{
			\begin{algorithm}[H]
				Wähle für die Zielverteilung $\pi$ bei $t=0$ die Startwerte $X_0$, $\mu_0$ und $\Sigma_0$ mit $\pi(X_0)>0$ sowie $\beta_{t+1} = 1/(t+1)$. Bestimme $X_{n}$ mit
				
				\While{$t < n$}{
				Ziehe $X^*$ aus $Q=\cN(X_t,\lambda\Sigma_t)$ und $\alpha$ aus $\cU([0,1])$.
				
				\eIf{$\pi(X^*)/\pi(X_{t})>\alpha$}{$X_{t+1} \gets X^*$}{$X_{t+1} \gets X_t$}
				$\mu_{t+1} \gets \mu_t + \beta_{t+1} (X_{t+1}-\mu_{t})$ 
				
				$\Sigma_{t+1} \gets \Sigma_t + \beta_{t+1} [(X_{t+1} - \mu_t)(X_{t+1} - \mu_t)^T-\Sigma_t]$
%				\begin{align*}
%					\mu_{t+1}&=\mu_t + \beta_{t+1} (X_{t+1}-\mu_{t+1}) \\ 
%					\Sigma_{t+1}&=\Sigma_t + \beta_{t+1} \left[(X_{t+1} - \mu_t)(X_{t+1} - \mu_t)^T-X_t\right]\\
%				\end{align*}
				}
			\end{algorithm}
			}
		\vspace{-0.1cm}
		\end{exampleblock}
    \end{frame}
	

	\begin{frame}[fragile]{Noch mehr Beispielcode}
			
			\begin{overprint}
				\onslide<1>
				\begin{minted}[breaklines, numbersep=3pt, gobble=4, fontsize=\footnotesize, framesep=2mm, tabsize=4]{julia}
				using LinearAlgebra # for eigenvalues
				using Pipe:@pipe # for cleaner code
				
				function Q(n, j, i) # variable part of the transition kernel
					return (n-j) in [-i, i] ? 1/2 : 0 # uniform distribution
				end
				
				function P(m, n, j, β) # assumes all values have probability > 0
					i = (m == n) ? 1 : 2
					ψ = (m == n == 1) | (m != n == 4) ? (1-β) : (1-β)/2
					return (j == n) ? β*1/3 + ψ : (1-β)*Q(n, j, i) + β*1/3
				end
				\end{minted}
				\onslide<2>
				\begin{minted}[breaklines, numbersep=3pt, gobble=4, fontsize=\footnotesize, framesep=2mm, tabsize=4]{julia}
				β = 1/2 # mixing parameter
				state = [1,3,4] # state space
				cartesian = Iterators.product(state, state) |>
					collect |> permutedims
				transition = zeros(9,9)
				# fill the transition matrix
				for n in 1:9
					for j in 1:9
						value_n = cartesian[n]
						value_j = cartesian[j]
						transition[n,j] = (value_n[2] == value_j[1]) ? 
						P(value_n[1], value_n[2], value_j[2], β) : 0
					end
				end
				\end{minted}
				\onslide<3>
				\begin{minted}[breaklines, numbersep=3pt, gobble=4, fontsize=\footnotesize, framesep=2mm, tabsize=4]{jlcon}
				julia> round.(transition, digits=2)
				9×9 Array{Float64,2}:
				0.67  0.17  0.17  0.0   0.0   0.0   0.0   0.0   0.0
				0.0   0.0   0.0   0.42  0.42  0.17  0.0   0.0   0.0
				0.0   0.0   0.0   0.0   0.0   0.0   0.17  0.17  0.67
				0.42  0.42  0.17  0.0   0.0   0.0   0.0   0.0   0.0
				0.0   0.0   0.0   0.17  0.42  0.42  0.0   0.0   0.0
				0.0   0.0   0.0   0.0   0.0   0.0   0.17  0.17  0.67
				0.42  0.42  0.17  0.0   0.0   0.0   0.0   0.0   0.0
				0.0   0.0   0.0   0.42  0.42  0.17  0.0   0.0   0.0
				0.0   0.0   0.0   0.0   0.0   0.0   0.17  0.42  0.42
				\end{minted}
				\onslide<4>
				\begin{minted}[breaklines, numbersep=3pt, gobble=4, fontsize=\footnotesize, framesep=2mm, tabsize=4]{julia}
				# find eigenvectors
				equilibrium = transpose(transition) |> eigvecs |> transpose
				# find eigenvector with eigenvalue 1
				eig = @pipe transpose(transition) |> eigvals |> 
					isapprox.(_, 1) |> findfirst(x -> x == true, _)
				μ = @pipe equilibrium[eig,:] |> 
					convert(Vector{Real}, _) |> transpose
				# normalize
				μ = μ/sum(μ)
				\end{minted}
				\begin{minted}[breaklines, numbersep=3pt, gobble=4, fontsize=\footnotesize, framesep=2mm, tabsize=4]{jlcon}
				julia> sum(μ[1:3]) # I use Julia btw
				0.35166240409207217
				\end{minted}
			\end{overprint}
	\end{frame}

	\section{Ausblick}
	\begin{frame}{Zusammenfassung}
		\begin{itemize}
			\item foo
		\end{itemize}
	\end{frame}

	\setbeamertemplate{frametitle continuation}{}
	\begin{frame}[allowframebreaks]{References}
		\printbibliography
	\end{frame}
\end{document}